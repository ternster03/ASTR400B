\documentclass{article}
\usepackage{graphicx} % Required for inserting images

\title{Homework 3 ASTR 400B}
\author{Tugg C Ernster}
\date{February 6 2025}


\usepackage{graphicx}
\begin{document}
\maketitle
\begin{table}[ht]
\begin{center}

% Increase font size 
\renewcommand{\arraystretch}{1.5}  % Increase row height
\resizebox{\textwidth}{!}{  % Resize table 
\Large  % Set font size 
\begin{tabular}{|l|l|l|l|l|l|}
% Hline makes lines between values
\hline
Galaxy Name & Halo mass & Disk mass & Bulge mass & Total Mass & fbar \\
\hline
Milky Way & 1.975 $\times 10^{12}$ M$_\odot$ & 0.075 $\times 10^{12}$ M$_\odot$ & 0.010 $\times 10^{12}$ M$_\odot$ & 2.060 $\times 10^{12}$ M$_\odot$ & 0.041 \\
\hline
M31 & 1.921 $\times 10^{12}$ M$_\odot$ & 0.120 $\times 10^{12}$ M$_\odot$ & 0.019 $\times 10^{12}$ M$_\odot$ & 2.060 $\times 10^{12}$ M$_\odot$ & 0.067 \\
\hline
M33 & 0.187 $\times 10^{12}$ M$_\odot$ & 0.009 $\times 10^{12}$ M$_\odot$ & 0.000 $\times 10^{12}$ M$_\odot$ & 0.196 $\times 10^{12}$ M$_\odot$ & 0.046 \\
\hline
Local Group & 4.083 $\times 10^{12}$ M$_\odot$ & 0.204 $\times 10^{12}$ M$_\odot$ & 0.029 $\times 10^{12}$ M$_\odot$ & 4.316 $\times 10^{12}$ M$_\odot$ & 0.052 \\
\hline
\end{tabular}
}
\end{center}
% Create caption for table
\caption{Galaxy Mass Table}
\end{table}

\maketitle
1. How does the total mass of the MW and M31 compare in this simulation? What galaxy component dominates this total mass? 
\newline
\newline
The total masses are the same and the galaxy component that dominates is the Halo mass or the dark matter mass.
\newline
\newline
2. How does the stellar mass of the MW and M31 compare? Which galaxy do you expect to be more luminous? 
\newline
\newline
The stellar mass of M31 is larger than that of MW. I would imagine that M31 would be more luminous because it has more baryonic matter or visible matter.
\newline
\newline
3. How does the total dark matter mass of MW and M31 compare in this simulation (ratio)? Is this surprising given their difference in stellar mass?  
\newline
\newline
The dark matter ratio of MW was greater than that of M31, no it is not surprising considering the stellar mass is greater in M31 than in in MW.
\newline
\newline
4. What is the ratio of stellar mass to total mass for each galaxy (i.e. the Baryon fraction)?
In the Universe, Ωb/Ωm ∼16% of all mass is locked up in baryons (gas & stars) vs.
dark matter. How does this ratio compare to the baryon fraction you computed for
each galaxy? Given that the total gas mass in the disks of these galaxies is negligible
compared to the stellar mass, any ideas for why the universal baryon fraction might
differ from that in these galaxies?
\newline
\newline
The ratio of stellar mass to total mass for the Milky Way is 0.041, for M31 it is 0.067, for M33 it is 0.046. This ratio is much larger for the Universe, meaning that there is much more baryonic matter in the Universe. The ratios might be different due to the large structures in the Universe, in galaxies there might be an over representation of dark matter in the Universe where as other structures might have an underrepresented abundance of dark matter, which averages out to 16 percent. 
\end{document}



